% Options for packages loaded elsewhere
\PassOptionsToPackage{unicode}{hyperref}
\PassOptionsToPackage{hyphens}{url}
%
\documentclass[
]{article}
\title{HDT1}
\author{AndresEmilioQ, Mirka M, Oscar de Leon}
\date{4/2/2022}

\usepackage{amsmath,amssymb}
\usepackage{lmodern}
\usepackage{iftex}
\ifPDFTeX
  \usepackage[T1]{fontenc}
  \usepackage[utf8]{inputenc}
  \usepackage{textcomp} % provide euro and other symbols
\else % if luatex or xetex
  \usepackage{unicode-math}
  \defaultfontfeatures{Scale=MatchLowercase}
  \defaultfontfeatures[\rmfamily]{Ligatures=TeX,Scale=1}
\fi
% Use upquote if available, for straight quotes in verbatim environments
\IfFileExists{upquote.sty}{\usepackage{upquote}}{}
\IfFileExists{microtype.sty}{% use microtype if available
  \usepackage[]{microtype}
  \UseMicrotypeSet[protrusion]{basicmath} % disable protrusion for tt fonts
}{}
\makeatletter
\@ifundefined{KOMAClassName}{% if non-KOMA class
  \IfFileExists{parskip.sty}{%
    \usepackage{parskip}
  }{% else
    \setlength{\parindent}{0pt}
    \setlength{\parskip}{6pt plus 2pt minus 1pt}}
}{% if KOMA class
  \KOMAoptions{parskip=half}}
\makeatother
\usepackage{xcolor}
\IfFileExists{xurl.sty}{\usepackage{xurl}}{} % add URL line breaks if available
\IfFileExists{bookmark.sty}{\usepackage{bookmark}}{\usepackage{hyperref}}
\hypersetup{
  pdftitle={HDT1},
  pdfauthor={AndresEmilioQ, Mirka M, Oscar de Leon},
  hidelinks,
  pdfcreator={LaTeX via pandoc}}
\urlstyle{same} % disable monospaced font for URLs
\usepackage[margin=1in]{geometry}
\usepackage{color}
\usepackage{fancyvrb}
\newcommand{\VerbBar}{|}
\newcommand{\VERB}{\Verb[commandchars=\\\{\}]}
\DefineVerbatimEnvironment{Highlighting}{Verbatim}{commandchars=\\\{\}}
% Add ',fontsize=\small' for more characters per line
\usepackage{framed}
\definecolor{shadecolor}{RGB}{248,248,248}
\newenvironment{Shaded}{\begin{snugshade}}{\end{snugshade}}
\newcommand{\AlertTok}[1]{\textcolor[rgb]{0.94,0.16,0.16}{#1}}
\newcommand{\AnnotationTok}[1]{\textcolor[rgb]{0.56,0.35,0.01}{\textbf{\textit{#1}}}}
\newcommand{\AttributeTok}[1]{\textcolor[rgb]{0.77,0.63,0.00}{#1}}
\newcommand{\BaseNTok}[1]{\textcolor[rgb]{0.00,0.00,0.81}{#1}}
\newcommand{\BuiltInTok}[1]{#1}
\newcommand{\CharTok}[1]{\textcolor[rgb]{0.31,0.60,0.02}{#1}}
\newcommand{\CommentTok}[1]{\textcolor[rgb]{0.56,0.35,0.01}{\textit{#1}}}
\newcommand{\CommentVarTok}[1]{\textcolor[rgb]{0.56,0.35,0.01}{\textbf{\textit{#1}}}}
\newcommand{\ConstantTok}[1]{\textcolor[rgb]{0.00,0.00,0.00}{#1}}
\newcommand{\ControlFlowTok}[1]{\textcolor[rgb]{0.13,0.29,0.53}{\textbf{#1}}}
\newcommand{\DataTypeTok}[1]{\textcolor[rgb]{0.13,0.29,0.53}{#1}}
\newcommand{\DecValTok}[1]{\textcolor[rgb]{0.00,0.00,0.81}{#1}}
\newcommand{\DocumentationTok}[1]{\textcolor[rgb]{0.56,0.35,0.01}{\textbf{\textit{#1}}}}
\newcommand{\ErrorTok}[1]{\textcolor[rgb]{0.64,0.00,0.00}{\textbf{#1}}}
\newcommand{\ExtensionTok}[1]{#1}
\newcommand{\FloatTok}[1]{\textcolor[rgb]{0.00,0.00,0.81}{#1}}
\newcommand{\FunctionTok}[1]{\textcolor[rgb]{0.00,0.00,0.00}{#1}}
\newcommand{\ImportTok}[1]{#1}
\newcommand{\InformationTok}[1]{\textcolor[rgb]{0.56,0.35,0.01}{\textbf{\textit{#1}}}}
\newcommand{\KeywordTok}[1]{\textcolor[rgb]{0.13,0.29,0.53}{\textbf{#1}}}
\newcommand{\NormalTok}[1]{#1}
\newcommand{\OperatorTok}[1]{\textcolor[rgb]{0.81,0.36,0.00}{\textbf{#1}}}
\newcommand{\OtherTok}[1]{\textcolor[rgb]{0.56,0.35,0.01}{#1}}
\newcommand{\PreprocessorTok}[1]{\textcolor[rgb]{0.56,0.35,0.01}{\textit{#1}}}
\newcommand{\RegionMarkerTok}[1]{#1}
\newcommand{\SpecialCharTok}[1]{\textcolor[rgb]{0.00,0.00,0.00}{#1}}
\newcommand{\SpecialStringTok}[1]{\textcolor[rgb]{0.31,0.60,0.02}{#1}}
\newcommand{\StringTok}[1]{\textcolor[rgb]{0.31,0.60,0.02}{#1}}
\newcommand{\VariableTok}[1]{\textcolor[rgb]{0.00,0.00,0.00}{#1}}
\newcommand{\VerbatimStringTok}[1]{\textcolor[rgb]{0.31,0.60,0.02}{#1}}
\newcommand{\WarningTok}[1]{\textcolor[rgb]{0.56,0.35,0.01}{\textbf{\textit{#1}}}}
\usepackage{graphicx}
\makeatletter
\def\maxwidth{\ifdim\Gin@nat@width>\linewidth\linewidth\else\Gin@nat@width\fi}
\def\maxheight{\ifdim\Gin@nat@height>\textheight\textheight\else\Gin@nat@height\fi}
\makeatother
% Scale images if necessary, so that they will not overflow the page
% margins by default, and it is still possible to overwrite the defaults
% using explicit options in \includegraphics[width, height, ...]{}
\setkeys{Gin}{width=\maxwidth,height=\maxheight,keepaspectratio}
% Set default figure placement to htbp
\makeatletter
\def\fps@figure{htbp}
\makeatother
\setlength{\emergencystretch}{3em} % prevent overfull lines
\providecommand{\tightlist}{%
  \setlength{\itemsep}{0pt}\setlength{\parskip}{0pt}}
\setcounter{secnumdepth}{-\maxdimen} % remove section numbering
\ifLuaTeX
  \usepackage{selnolig}  % disable illegal ligatures
\fi

\begin{document}
\maketitle

\begin{Shaded}
\begin{Highlighting}[]
\FunctionTok{library}\NormalTok{(dplyr)}
\end{Highlighting}
\end{Shaded}

\begin{verbatim}
## 
## Attaching package: 'dplyr'
\end{verbatim}

\begin{verbatim}
## The following objects are masked from 'package:stats':
## 
##     filter, lag
\end{verbatim}

\begin{verbatim}
## The following objects are masked from 'package:base':
## 
##     intersect, setdiff, setequal, union
\end{verbatim}

\begin{Shaded}
\begin{Highlighting}[]
\NormalTok{datos }\OtherTok{\textless{}{-}} \FunctionTok{read.csv}\NormalTok{(}\StringTok{"movies.csv"}\NormalTok{)}
\end{Highlighting}
\end{Shaded}

\hypertarget{a-continuacion-los-datos-recopilados-de-la-base-de-datos-y-sus-caracteristicas}{%
\subsection{A continuacion los datos recopilados de la base de datos y
sus
caracteristicas}\label{a-continuacion-los-datos-recopilados-de-la-base-de-datos-y-sus-caracteristicas}}

\hypertarget{el-conjunto-de-datos-esta-compuesto-por-10000-observaciones-y-27-variables.}{%
\subsection{El conjunto de datos esta compuesto por 10000 observaciones
y 27
variables.}\label{el-conjunto-de-datos-esta-compuesto-por-10000-observaciones-y-27-variables.}}

\hypertarget{diga-el-tipo-de-cada-una-de-las-variables-cualitativa-ordinal-o-nominal-cuantitativa}{%
\subsection{2. Diga el tipo de cada una de las variables (cualitativa
ordinal o nominal,
cuantitativa}\label{diga-el-tipo-de-cada-una-de-las-variables-cualitativa-ordinal-o-nominal-cuantitativa}}

continua, cuantitativa discreta)

\begin{Shaded}
\begin{Highlighting}[]
\NormalTok{variable }\OtherTok{\textless{}{-}} \FunctionTok{c}\NormalTok{(}\StringTok{"id"}\NormalTok{, }\StringTok{"popularity"}\NormalTok{, }\StringTok{"budget"}\NormalTok{, }\StringTok{"revenue"}\NormalTok{, }\StringTok{"original\_title"}\NormalTok{, }\StringTok{"originalLanguage"}\NormalTok{, }\StringTok{"title"}\NormalTok{, }\StringTok{"homePage"}\NormalTok{, }\StringTok{"video"}\NormalTok{, }\StringTok{"director"}\NormalTok{, }\StringTok{"runtime"}\NormalTok{, }\StringTok{"genres"}\NormalTok{, }\StringTok{"genresAmount"}\NormalTok{, }\StringTok{"productionCompany"}\NormalTok{, }\StringTok{"productionCoAmount"}\NormalTok{, }\StringTok{"productionCompanyCountry"}\NormalTok{, }\StringTok{"productionCountry"}\NormalTok{, }\StringTok{"productionCountriesAmount"}\NormalTok{, }\StringTok{"releaseDate"}\NormalTok{, }\StringTok{"voteCount"}\NormalTok{, }\StringTok{"voteAvg"}\NormalTok{, }\StringTok{"actors"}\NormalTok{, }\StringTok{"actorsPopularity"}\NormalTok{, }\StringTok{"actorsCharacter"}\NormalTok{, }\StringTok{"actorsAmount"}\NormalTok{, }\StringTok{"castWomenAmount"}\NormalTok{, }\StringTok{"CastMenAmount"}\NormalTok{)}

\NormalTok{tipo }\OtherTok{\textless{}{-}} \FunctionTok{c}\NormalTok{(}\StringTok{"cuantitativa continua"}\NormalTok{, }\StringTok{"cuantitativa continua"}\NormalTok{, }\StringTok{"cuantitativa continua "}\NormalTok{, }\StringTok{"cuantitativa continua"}\NormalTok{, }\StringTok{"cualitativa nominal"}\NormalTok{, }\StringTok{"cualitativa nominal"}\NormalTok{, }\StringTok{"cualitativa nominal"}\NormalTok{, }\StringTok{"cualitativa nominal"}\NormalTok{, }\StringTok{"cualitativa nominal"}\NormalTok{, }\StringTok{"cualitativa nominal"}\NormalTok{, }\StringTok{"cuantitativa continua"}\NormalTok{, }\StringTok{"cualitativa nominal"}\NormalTok{, }\StringTok{"cuantitativa discreta"}\NormalTok{, }\StringTok{"cualitativa nominal"}\NormalTok{, }\StringTok{"cuantitativa discreta"}\NormalTok{, }\StringTok{"cualitativa nominal"}\NormalTok{, }\StringTok{"cualitativa nominal"}\NormalTok{, }\StringTok{" cuantitativa discreta"}\NormalTok{, }\StringTok{"cualitativa nominal"}\NormalTok{, }\StringTok{"cualitativa discreta"}\NormalTok{, }\StringTok{"cuantitativa continua"}\NormalTok{, }\StringTok{"cualitativa nominal"}\NormalTok{, }\StringTok{"cuantitativa continua"}\NormalTok{, }\StringTok{"cualitativa nominal"}\NormalTok{, }\StringTok{"cuantitativa discreta"}\NormalTok{, }\StringTok{"cuantitativa discreta"}\NormalTok{, }\StringTok{"cuantitativa discreta"}\NormalTok{)}

\NormalTok{numero }\OtherTok{\textless{}{-}} \DecValTok{1}\SpecialCharTok{:}\DecValTok{27}

\NormalTok{DataFrame.Variables }\OtherTok{\textless{}{-}} \FunctionTok{data.frame}\NormalTok{(numero, variable, tipo)}

\FunctionTok{print}\NormalTok{(DataFrame.Variables)}
\end{Highlighting}
\end{Shaded}

\begin{verbatim}
##    numero                  variable                   tipo
## 1       1                        id  cuantitativa continua
## 2       2                popularity  cuantitativa continua
## 3       3                    budget cuantitativa continua 
## 4       4                   revenue  cuantitativa continua
## 5       5            original_title    cualitativa nominal
## 6       6          originalLanguage    cualitativa nominal
## 7       7                     title    cualitativa nominal
## 8       8                  homePage    cualitativa nominal
## 9       9                     video    cualitativa nominal
## 10     10                  director    cualitativa nominal
## 11     11                   runtime  cuantitativa continua
## 12     12                    genres    cualitativa nominal
## 13     13              genresAmount  cuantitativa discreta
## 14     14         productionCompany    cualitativa nominal
## 15     15        productionCoAmount  cuantitativa discreta
## 16     16  productionCompanyCountry    cualitativa nominal
## 17     17         productionCountry    cualitativa nominal
## 18     18 productionCountriesAmount  cuantitativa discreta
## 19     19               releaseDate    cualitativa nominal
## 20     20                 voteCount   cualitativa discreta
## 21     21                   voteAvg  cuantitativa continua
## 22     22                    actors    cualitativa nominal
## 23     23          actorsPopularity  cuantitativa continua
## 24     24           actorsCharacter    cualitativa nominal
## 25     25              actorsAmount  cuantitativa discreta
## 26     26           castWomenAmount  cuantitativa discreta
## 27     27             CastMenAmount  cuantitativa discreta
\end{verbatim}

\hypertarget{investigue-si-las-variables-cuantitativas-siguen-una-distribuciuxf3n-normal-y-haga-una}{%
\subsection{3. Investigue si las variables cuantitativas siguen una
distribución normal y haga
una}\label{investigue-si-las-variables-cuantitativas-siguen-una-distribuciuxf3n-normal-y-haga-una}}

tabla de frecuencias de las variables cualitativas. Explique todos los
resultados.

Nos apoyamos de nuestros conocimientos previos en estadistica y
utilizamos ademas la curtosis para determinar la concentracion de los
datos entorno a la media. Encontramos que todas nuestras variables
cuantitativas obtuvieron un coeficiente mayor a uno, es decir positivo
(distribucion leptocúrtica)

Referencia y expliacion detallada:
\url{https://economipedia.com/definiciones/curtosis.html}

\begin{Shaded}
\begin{Highlighting}[]
\CommentTok{\#Libreria necesaria para utilizar la curtosis, por favor instalar}
\FunctionTok{library}\NormalTok{(e1071)}

\DocumentationTok{\#\# Popularidad}

\NormalTok{popularity }\OtherTok{=}\NormalTok{ datos}\SpecialCharTok{$}\NormalTok{popularity}
\StringTok{"Kurtosis de popularidad"}
\end{Highlighting}
\end{Shaded}

\begin{verbatim}
## [1] "Kurtosis de popularidad"
\end{verbatim}

\begin{Shaded}
\begin{Highlighting}[]
\CommentTok{\#Presupuesto}
\NormalTok{budget }\OtherTok{=}\NormalTok{ datos}\SpecialCharTok{$}\NormalTok{budget}
\StringTok{"Kurtosis de budget"}
\end{Highlighting}
\end{Shaded}

\begin{verbatim}
## [1] "Kurtosis de budget"
\end{verbatim}

\begin{Shaded}
\begin{Highlighting}[]
\FunctionTok{kurtosis}\NormalTok{(budget)}
\end{Highlighting}
\end{Shaded}

\begin{verbatim}
## [1] 13.20779
\end{verbatim}

\begin{Shaded}
\begin{Highlighting}[]
\CommentTok{\#Ingresos}
\NormalTok{revenue }\OtherTok{=}\NormalTok{ datos}\SpecialCharTok{$}\NormalTok{revenue}
\StringTok{"Kurtosis de revenue"}
\end{Highlighting}
\end{Shaded}

\begin{verbatim}
## [1] "Kurtosis de revenue"
\end{verbatim}

\begin{Shaded}
\begin{Highlighting}[]
\FunctionTok{kurtosis}\NormalTok{(revenue)}
\end{Highlighting}
\end{Shaded}

\begin{verbatim}
## [1] 55.89168
\end{verbatim}

\begin{Shaded}
\begin{Highlighting}[]
\CommentTok{\#Duración}
\NormalTok{runtime }\OtherTok{=}\NormalTok{ datos}\SpecialCharTok{$}\NormalTok{runtime}
\StringTok{"Kurtosis de runtime"}
\end{Highlighting}
\end{Shaded}

\begin{verbatim}
## [1] "Kurtosis de runtime"
\end{verbatim}

\begin{Shaded}
\begin{Highlighting}[]
\FunctionTok{kurtosis}\NormalTok{(runtime)}
\end{Highlighting}
\end{Shaded}

\begin{verbatim}
## [1] 35.52866
\end{verbatim}

\begin{Shaded}
\begin{Highlighting}[]
\CommentTok{\#Votos}
\NormalTok{vote\_count }\OtherTok{=}\NormalTok{ datos}\SpecialCharTok{$}\NormalTok{voteCount}
\StringTok{"Kurtosis de VoteCount"}
\end{Highlighting}
\end{Shaded}

\begin{verbatim}
## [1] "Kurtosis de VoteCount"
\end{verbatim}

\begin{Shaded}
\begin{Highlighting}[]
\FunctionTok{kurtosis}\NormalTok{(vote\_count)}
\end{Highlighting}
\end{Shaded}

\begin{verbatim}
## [1] 22.76501
\end{verbatim}

\begin{Shaded}
\begin{Highlighting}[]
\CommentTok{\#Promedio de votos}
\NormalTok{vote\_average }\OtherTok{=}\NormalTok{ datos}\SpecialCharTok{$}\NormalTok{voteAvg}
\StringTok{"Kurtosis de VoteAvg"}
\end{Highlighting}
\end{Shaded}

\begin{verbatim}
## [1] "Kurtosis de VoteAvg"
\end{verbatim}

\begin{Shaded}
\begin{Highlighting}[]
\FunctionTok{kurtosis}\NormalTok{(vote\_average)}
\end{Highlighting}
\end{Shaded}

\begin{verbatim}
## [1] 1.484826
\end{verbatim}

\begin{Shaded}
\begin{Highlighting}[]
\CommentTok{\#Cantidad de generos}
\NormalTok{GenresA }\OtherTok{=}\NormalTok{ datos}\SpecialCharTok{$}\NormalTok{genresAmount}
\StringTok{"Kurtosis de GenresAmount"}
\end{Highlighting}
\end{Shaded}

\begin{verbatim}
## [1] "Kurtosis de GenresAmount"
\end{verbatim}

\begin{Shaded}
\begin{Highlighting}[]
\FunctionTok{kurtosis}\NormalTok{(GenresA)}
\end{Highlighting}
\end{Shaded}

\begin{verbatim}
## [1] 2.104824
\end{verbatim}

\begin{Shaded}
\begin{Highlighting}[]
\CommentTok{\#Cantidad de co productora }
\NormalTok{ProductionCoA }\OtherTok{=}\NormalTok{ datos}\SpecialCharTok{$}\NormalTok{productionCoAmount}
\StringTok{"Kurtosis de productionCoAmount"}
\end{Highlighting}
\end{Shaded}

\begin{verbatim}
## [1] "Kurtosis de productionCoAmount"
\end{verbatim}

\begin{Shaded}
\begin{Highlighting}[]
\FunctionTok{kurtosis}\NormalTok{(ProductionCoA)}
\end{Highlighting}
\end{Shaded}

\begin{verbatim}
## [1] 140.2735
\end{verbatim}

\begin{Shaded}
\begin{Highlighting}[]
\CommentTok{\#Cantidad de productoras por pais}
\NormalTok{ProducCountrieA }\OtherTok{=}\NormalTok{ datos}\SpecialCharTok{$}\NormalTok{productionCountriesAmount}
\StringTok{"Kurtosis de ProductCountriesAmount"}
\end{Highlighting}
\end{Shaded}

\begin{verbatim}
## [1] "Kurtosis de ProductCountriesAmount"
\end{verbatim}

\begin{Shaded}
\begin{Highlighting}[]
\FunctionTok{kurtosis}\NormalTok{(ProducCountrieA)}
\end{Highlighting}
\end{Shaded}

\begin{verbatim}
## [1] 751.2015
\end{verbatim}

\begin{Shaded}
\begin{Highlighting}[]
\CommentTok{\#Cantidad de actores}
\NormalTok{ActorsA }\OtherTok{=}\NormalTok{ datos}\SpecialCharTok{$}\NormalTok{actorsAmount}
\StringTok{"Kurtosis de ActorsAmount"}
\end{Highlighting}
\end{Shaded}

\begin{verbatim}
## [1] "Kurtosis de ActorsAmount"
\end{verbatim}

\begin{Shaded}
\begin{Highlighting}[]
\FunctionTok{kurtosis}\NormalTok{(ActorsA)}
\end{Highlighting}
\end{Shaded}

\begin{verbatim}
## [1] 371.0959
\end{verbatim}

\begin{Shaded}
\begin{Highlighting}[]
\CommentTok{\#Cantiddad de actores Masculinos}
\CommentTok{\#Esta kurtosis la realizamos en excel debido a que la base de datos presenta datos no numericos en ciertas celdas}
\NormalTok{ActoresM }\OtherTok{=}\NormalTok{ datos}\SpecialCharTok{$}\NormalTok{castMenAmount}
\NormalTok{ActoresM }\OtherTok{\textless{}{-}} \FunctionTok{na.omit}\NormalTok{(ActoresM)}
\NormalTok{b }\OtherTok{\textless{}{-}} \FloatTok{43.93458}
\StringTok{"Kurtosis de CastMenAmount"}
\end{Highlighting}
\end{Shaded}

\begin{verbatim}
## [1] "Kurtosis de CastMenAmount"
\end{verbatim}

\begin{Shaded}
\begin{Highlighting}[]
\NormalTok{b}
\end{Highlighting}
\end{Shaded}

\begin{verbatim}
## [1] 43.93458
\end{verbatim}

\begin{Shaded}
\begin{Highlighting}[]
\CommentTok{\#kurtosis(ActoresM)}

\CommentTok{\#Cantiddad de actrices Femenina}
\CommentTok{\#Esta kurtosis la realizamos en excel debido a que la base de datos presenta datos no numericos en ciertas celdas}
\NormalTok{ActricesF }\OtherTok{=}\NormalTok{ datos}\SpecialCharTok{$}\NormalTok{castWomenAmount}
\NormalTok{a }\OtherTok{\textless{}{-}} \FloatTok{108.7062}
\StringTok{"Kurtosis de CastWomenAmount"}
\end{Highlighting}
\end{Shaded}

\begin{verbatim}
## [1] "Kurtosis de CastWomenAmount"
\end{verbatim}

\begin{Shaded}
\begin{Highlighting}[]
\NormalTok{a}
\end{Highlighting}
\end{Shaded}

\begin{verbatim}
## [1] 108.7062
\end{verbatim}

\begin{Shaded}
\begin{Highlighting}[]
\CommentTok{\#kurtosis(ActricesF)}
\end{Highlighting}
\end{Shaded}

\hypertarget{variables-cualitativas-tabla-de-frecuencias}{%
\subparagraph{Variables cualitativas tabla de
frecuencias}\label{variables-cualitativas-tabla-de-frecuencias}}

Para las variables cualitativas realizamos tablas de frecuencia con la
ayuda de data frames para poder observar y analizar los datos de mejor
manera. Se ignoro la variable id debido a que se determino a que no
aporta al analisis estadistico. Se noto que distintas variables contaban
con valores vacios, se llenaron con NA y otros varios estaban
acompañados por slash o pipe, por lo cual se decidio hacer un split para
estos terminos.

\begin{Shaded}
\begin{Highlighting}[]
\FunctionTok{library}\NormalTok{(tidyr)}
\CommentTok{\#Generos}
\NormalTok{genres }\OtherTok{\textless{}{-}} \FunctionTok{data.frame}\NormalTok{(}\FunctionTok{table}\NormalTok{(}\FunctionTok{do.call}\NormalTok{(c, }\FunctionTok{lapply}\NormalTok{(datos}\SpecialCharTok{$}\NormalTok{genres, }\ControlFlowTok{function}\NormalTok{(x) }\FunctionTok{unlist}\NormalTok{(}\FunctionTok{strsplit}\NormalTok{(x, }\StringTok{"}\SpecialCharTok{\textbackslash{}\textbackslash{}}\StringTok{|"}\NormalTok{))))))}

\CommentTok{\#Pagina}
\NormalTok{homepage }\OtherTok{\textless{}{-}} \FunctionTok{data.frame}\NormalTok{(}\FunctionTok{table}\NormalTok{(datos}\SpecialCharTok{$}\NormalTok{homePage))}
\NormalTok{homepage[homepage }\SpecialCharTok{==} \StringTok{""}\NormalTok{] }\OtherTok{\textless{}{-}} \ConstantTok{NA}
\NormalTok{homepage }\OtherTok{\textless{}{-}} \FunctionTok{na.omit}\NormalTok{(homepage)}

\CommentTok{\#Companias de produccion}
\NormalTok{productionCompany }\OtherTok{\textless{}{-}} \FunctionTok{data.frame}\NormalTok{(}\FunctionTok{table}\NormalTok{(}\FunctionTok{do.call}\NormalTok{(c, }\FunctionTok{lapply}\NormalTok{(datos}\SpecialCharTok{$}\NormalTok{productionCompany, }\ControlFlowTok{function}\NormalTok{(x) }\FunctionTok{unlist}\NormalTok{(}\FunctionTok{strsplit}\NormalTok{(x, }\StringTok{"}\SpecialCharTok{\textbackslash{}\textbackslash{}}\StringTok{|"}\NormalTok{))))))}

\CommentTok{\#pais de compania de produccion}
\NormalTok{ProductionCompanyCountry }\OtherTok{\textless{}{-}} \FunctionTok{data.frame}\NormalTok{(}\FunctionTok{table}\NormalTok{(}\FunctionTok{do.call}\NormalTok{(c, }\FunctionTok{lapply}\NormalTok{(datos}\SpecialCharTok{$}\NormalTok{productionCompanyCountry, }\ControlFlowTok{function}\NormalTok{(x) }\FunctionTok{unlist}\NormalTok{(}\FunctionTok{strsplit}\NormalTok{(x, }\StringTok{"}\SpecialCharTok{\textbackslash{}\textbackslash{}}\StringTok{|"}\NormalTok{))))))}

\CommentTok{\#pais productora}
\NormalTok{ProductionCountry }\OtherTok{\textless{}{-}} \FunctionTok{data.frame}\NormalTok{(}\FunctionTok{table}\NormalTok{(}\FunctionTok{do.call}\NormalTok{(c, }\FunctionTok{lapply}\NormalTok{(datos}\SpecialCharTok{$}\NormalTok{productionCountry, }\ControlFlowTok{function}\NormalTok{(x) }\FunctionTok{unlist}\NormalTok{(}\FunctionTok{strsplit}\NormalTok{(x, }\StringTok{"}\SpecialCharTok{\textbackslash{}\textbackslash{}}\StringTok{|"}\NormalTok{))))))}

\CommentTok{\#video}
\NormalTok{video }\OtherTok{\textless{}{-}} \FunctionTok{data.frame}\NormalTok{(}\FunctionTok{table}\NormalTok{(datos}\SpecialCharTok{$}\NormalTok{video))}
\NormalTok{video[video }\SpecialCharTok{==} \StringTok{""}\NormalTok{] }\OtherTok{\textless{}{-}} \ConstantTok{NA}
\NormalTok{video }\OtherTok{\textless{}{-}} \FunctionTok{na.omit}\NormalTok{(video)}

\CommentTok{\#Director}
\NormalTok{director }\OtherTok{\textless{}{-}} \FunctionTok{data.frame}\NormalTok{(}\FunctionTok{table}\NormalTok{(datos}\SpecialCharTok{$}\NormalTok{director))}
\NormalTok{director[director }\SpecialCharTok{==} \StringTok{""}\NormalTok{] }\OtherTok{\textless{}{-}} \ConstantTok{NA}
\NormalTok{director }\OtherTok{\textless{}{-}} \FunctionTok{na.omit}\NormalTok{(director)}

\CommentTok{\#Actores}
\NormalTok{actors }\OtherTok{\textless{}{-}} \FunctionTok{data.frame}\NormalTok{(}\FunctionTok{table}\NormalTok{(}\FunctionTok{do.call}\NormalTok{(c, }\FunctionTok{lapply}\NormalTok{(datos}\SpecialCharTok{$}\NormalTok{actors, }\ControlFlowTok{function}\NormalTok{(x) }\FunctionTok{unlist}\NormalTok{(}\FunctionTok{strsplit}\NormalTok{(x, }\StringTok{"}\SpecialCharTok{\textbackslash{}\textbackslash{}}\StringTok{|"}\NormalTok{))))))}
\NormalTok{actors }\OtherTok{\textless{}{-}} \FunctionTok{data.frame}\NormalTok{(}\FunctionTok{table}\NormalTok{(datos}\SpecialCharTok{$}\NormalTok{actors))}
\NormalTok{actors[actors }\SpecialCharTok{==} \StringTok{""}\NormalTok{] }\OtherTok{\textless{}{-}} \ConstantTok{NA}
\NormalTok{actors }\OtherTok{\textless{}{-}} \FunctionTok{na.omit}\NormalTok{(actors)}

\CommentTok{\#ActorsCharacter}
\NormalTok{character }\OtherTok{\textless{}{-}} \FunctionTok{data.frame}\NormalTok{(}\FunctionTok{table}\NormalTok{(}\FunctionTok{do.call}\NormalTok{(c, }\FunctionTok{lapply}\NormalTok{(datos}\SpecialCharTok{$}\NormalTok{actorsCharacter, }\ControlFlowTok{function}\NormalTok{(x) }\FunctionTok{unlist}\NormalTok{(}\FunctionTok{strsplit}\NormalTok{(x, }\StringTok{"}\SpecialCharTok{\textbackslash{}\textbackslash{}}\StringTok{|"}\NormalTok{))))))}
\NormalTok{character[character }\SpecialCharTok{==} \StringTok{""}\NormalTok{] }\OtherTok{\textless{}{-}} \ConstantTok{NA}
\NormalTok{character }\OtherTok{\textless{}{-}} \FunctionTok{na.omit}\NormalTok{(character)}

\CommentTok{\#Titulo original}
\NormalTok{original\_title }\OtherTok{\textless{}{-}} \FunctionTok{data.frame}\NormalTok{(}\FunctionTok{table}\NormalTok{(datos}\SpecialCharTok{$}\NormalTok{originalTitle))}

\CommentTok{\#Titulo}
\NormalTok{title }\OtherTok{\textless{}{-}} \FunctionTok{data.frame}\NormalTok{(}\FunctionTok{table}\NormalTok{(datos}\SpecialCharTok{$}\NormalTok{title))}

\CommentTok{\#Lenguaje Original}
\NormalTok{language }\OtherTok{\textless{}{-}} \FunctionTok{data.frame}\NormalTok{(}\FunctionTok{table}\NormalTok{(datos}\SpecialCharTok{$}\NormalTok{originalLanguage))}

\CommentTok{\#fecha}
\NormalTok{release\_date }\OtherTok{\textless{}{-}} \FunctionTok{data.frame}\NormalTok{(}\FunctionTok{table}\NormalTok{(datos}\SpecialCharTok{$}\NormalTok{releaseDate))}
\end{Highlighting}
\end{Shaded}

\hypertarget{responda-las-siguientes-preguntas}{%
\subsection{4. Responda las siguientes
preguntas:}\label{responda-las-siguientes-preguntas}}

\hypertarget{cuuxe1l-es-la-peor-peluxedcula-de-acuerdo-a-los-votos-de-todos-los-usuarios}{%
\subsubsection{4.4 ¿Cuál es la peor película de acuerdo a los votos de
todos los
usuarios?}\label{cuuxe1l-es-la-peor-peluxedcula-de-acuerdo-a-los-votos-de-todos-los-usuarios}}

A continuacion las dos peor votadas

\begin{Shaded}
\begin{Highlighting}[]
\CommentTok{\# peorvotadas \textless{}{-} datos[order(datos$voteAvg),]}
\CommentTok{\#peorvotadas \textless{}{-} peorvotadas[1,]}
\CommentTok{\#na.omit(peorvotadas)}
\NormalTok{peorvotadas }\OtherTok{\textless{}{-}} \FunctionTok{data.frame}\NormalTok{(datos}\SpecialCharTok{$}\NormalTok{voteAvg, datos}\SpecialCharTok{$}\NormalTok{title)}
\NormalTok{lapeor }\OtherTok{\textless{}{-}} \FunctionTok{data.frame}\NormalTok{(peorvotadas[}\FunctionTok{order}\NormalTok{(peorvotadas}\SpecialCharTok{$}\NormalTok{datos.voteAvg),])}
\FunctionTok{head}\NormalTok{(lapeor,}\DecValTok{2}\NormalTok{)}
\end{Highlighting}
\end{Shaded}

\begin{verbatim}
##      datos.voteAvg
## 9787           1.3
## 9707           1.5
##                                                                           datos.title
## 9787 DAKAICHI -I'm Being Harassed by the Sexiest Man of the Year- The Movie: In Spain
## 9707                                                DRagON BALL P2 2wice dropda bbeet
\end{verbatim}

\hypertarget{cuuxe1ntas-peluxedculas-se-hicieron-en-cada-auxf1o-en-quuxe9-auxf1o-se-hicieron-muxe1s-peluxedculas-haga-un-gruxe1fico-de-barras}{%
\subsubsection{4.5 ¿Cuántas películas se hicieron en cada año? ¿En qué
año se hicieron más películas? Haga un gráfico de
barras}\label{cuuxe1ntas-peluxedculas-se-hicieron-en-cada-auxf1o-en-quuxe9-auxf1o-se-hicieron-muxe1s-peluxedculas-haga-un-gruxe1fico-de-barras}}

\begin{Shaded}
\begin{Highlighting}[]
\NormalTok{Dates }\OtherTok{\textless{}{-}} \FunctionTok{data.frame}\NormalTok{(datos}\SpecialCharTok{$}\NormalTok{title, datos}\SpecialCharTok{$}\NormalTok{releaseDate)}
\NormalTok{PorAnioDesc }\OtherTok{\textless{}{-}} \FunctionTok{data.frame}\NormalTok{(Dates[}\FunctionTok{order}\NormalTok{(Dates}\SpecialCharTok{$}\NormalTok{datos.releaseDate),])}
\NormalTok{year1 }\OtherTok{\textless{}{-}} \FunctionTok{data.frame}\NormalTok{(}\FunctionTok{table}\NormalTok{(}\FunctionTok{substring}\NormalTok{(PorAnioDesc}\SpecialCharTok{$}\NormalTok{datos.releaseDate,}\DecValTok{1}\NormalTok{,}\DecValTok{4}\NormalTok{)))}
\NormalTok{finales }\OtherTok{\textless{}{-}} \FunctionTok{as.numeric}\NormalTok{(}\FunctionTok{substring}\NormalTok{(PorAnioDesc}\SpecialCharTok{$}\NormalTok{datos.releaseDate,}\DecValTok{1}\NormalTok{,}\DecValTok{4}\NormalTok{))}
\FunctionTok{hist}\NormalTok{(finales,}\AttributeTok{breaks =} \DecValTok{100}\NormalTok{)}
\end{Highlighting}
\end{Shaded}

\includegraphics{AnalisisMineria1_files/figure-latex/unnamed-chunk-6-1.pdf}

\hypertarget{cuuxe1l-es-el-guxe9nero-principal-de-las-20-peluxedculas-muxe1s-recientes-cuuxe1l-es-el-guxe9nero-principal-que-predomina-en-el-conjunto-de-datos-represuxe9ntelo-usando-un-gruxe1fico}{%
\subsubsection{4.6 ¿Cuál es el género principal de las 20 películas más
recientes? ¿Cuál es el género principal que predomina en el conjunto de
datos? Represéntelo usando un
gráfico}\label{cuuxe1l-es-el-guxe9nero-principal-de-las-20-peluxedculas-muxe1s-recientes-cuuxe1l-es-el-guxe9nero-principal-que-predomina-en-el-conjunto-de-datos-represuxe9ntelo-usando-un-gruxe1fico}}

A continuacion el top20:

\begin{Shaded}
\begin{Highlighting}[]
\NormalTok{LicaPopular }\OtherTok{\textless{}{-}}\NormalTok{ datos[}\FunctionTok{order}\NormalTok{(datos}\SpecialCharTok{$}\NormalTok{popularity,}\AttributeTok{decreasing =} \ConstantTok{TRUE}\NormalTok{),]}
\NormalTok{top20 }\OtherTok{\textless{}{-}}\NormalTok{ LicaPopular[}\DecValTok{1}\SpecialCharTok{:}\DecValTok{20}\NormalTok{,}\FunctionTok{c}\NormalTok{(}\StringTok{"genres"}\NormalTok{)]}
\NormalTok{genres20 }\OtherTok{\textless{}{-}} \FunctionTok{unlist}\NormalTok{(}\FunctionTok{strsplit}\NormalTok{(}\FunctionTok{as.character}\NormalTok{(top20), }\StringTok{"}\SpecialCharTok{\textbackslash{}\textbackslash{}}\StringTok{|"}\NormalTok{))}
\NormalTok{genres20}
\end{Highlighting}
\end{Shaded}

\begin{verbatim}
##  [1] "Action"          "Adventure"       "Fantasy"         "Science Fiction"
##  [5] "Action"          "Adventure"       "Science Fiction" "Animation"      
##  [9] "Comedy"          "Family"          "Music"           "Horror"         
## [13] "Action"          "Science Fiction" "Animation"       "Comedy"         
## [17] "Family"          "Fantasy"         "Comedy"          "Fantasy"        
## [21] "Adventure"       "Action"          "Thriller"        "Science Fiction"
## [25] "Action"          "Adventure"       "Science Fiction" "Action"         
## [29] "Adventure"       "Science Fiction" "Thriller"        "Action"         
## [33] "Thriller"        "Action"          "Comedy"          "Crime"          
## [37] "Thriller"        "Action"          "Adventure"       "Fantasy"        
## [41] "Action"          "Thriller"        "Crime"           "Drama"          
## [45] "Horror"          "Mystery"         "Animation"       "Comedy"         
## [49] "Family"          "Crime"           "Action"          "Thriller"       
## [53] "Animation"       "Action"          "Adventure"       "Fantasy"        
## [57] "Drama"           "History"         "Adventure"       "Horror"         
## [61] "Thriller"
\end{verbatim}

Para encontrar el genero principal predominante debemos separar la
lista, debido a que las peliculas cuentan con mas de un genero.

\begin{Shaded}
\begin{Highlighting}[]
\NormalTok{getmode }\OtherTok{\textless{}{-}} \ControlFlowTok{function}\NormalTok{(v) \{}
\NormalTok{  uniqv }\OtherTok{\textless{}{-}} \FunctionTok{unique}\NormalTok{(v)}
\NormalTok{  uniqv[}\FunctionTok{which.max}\NormalTok{(}\FunctionTok{tabulate}\NormalTok{(}\FunctionTok{match}\NormalTok{(v, uniqv)))]}
\NormalTok{\}}
\NormalTok{topGenre20 }\OtherTok{\textless{}{-}} \FunctionTok{getmode}\NormalTok{(genres20)}
\StringTok{"Se encontro que el genero principal de las peliculas es:"}
\end{Highlighting}
\end{Shaded}

\begin{verbatim}
## [1] "Se encontro que el genero principal de las peliculas es:"
\end{verbatim}

\begin{Shaded}
\begin{Highlighting}[]
\NormalTok{topGenre20}
\end{Highlighting}
\end{Shaded}

\begin{verbatim}
## [1] "Action"
\end{verbatim}

\begin{Shaded}
\begin{Highlighting}[]
\NormalTok{totalGenres }\OtherTok{\textless{}{-}} \FunctionTok{unlist}\NormalTok{(}\FunctionTok{strsplit}\NormalTok{(}\FunctionTok{as.character}\NormalTok{(datos}\SpecialCharTok{$}\NormalTok{genres), }\StringTok{"}\SpecialCharTok{\textbackslash{}\textbackslash{}}\StringTok{|"}\NormalTok{))}
\FunctionTok{barplot}\NormalTok{(}\FunctionTok{table}\NormalTok{(totalGenres))}
\end{Highlighting}
\end{Shaded}

\includegraphics{AnalisisMineria1_files/figure-latex/unnamed-chunk-8-1.pdf}

\hypertarget{las-peluxedculas-de-quuxe9-genero-principal-obtuvieron-mayores-ganancias}{%
\subsubsection{4.7 ¿Las películas de qué genero principal obtuvieron
mayores
ganancias?}\label{las-peluxedculas-de-quuxe9-genero-principal-obtuvieron-mayores-ganancias}}

\begin{Shaded}
\begin{Highlighting}[]
\NormalTok{worth }\OtherTok{\textless{}{-}} \FunctionTok{data.frame}\NormalTok{(datos}\SpecialCharTok{$}\NormalTok{revenue, datos}\SpecialCharTok{$}\NormalTok{genres, datos}\SpecialCharTok{$}\NormalTok{title)}
\NormalTok{ordenadas }\OtherTok{\textless{}{-}} \FunctionTok{data.frame}\NormalTok{(worth[}\FunctionTok{order}\NormalTok{(}\SpecialCharTok{{-}}\NormalTok{datos}\SpecialCharTok{$}\NormalTok{revenue), ])}
\StringTok{"Las peliculas del genero de accion fueron las que obtuvieron mayores ganancias"}
\end{Highlighting}
\end{Shaded}

\begin{verbatim}
## [1] "Las peliculas del genero de accion fueron las que obtuvieron mayores ganancias"
\end{verbatim}

\begin{Shaded}
\begin{Highlighting}[]
\FunctionTok{head}\NormalTok{(ordenadas, }\DecValTok{5}\NormalTok{)}
\end{Highlighting}
\end{Shaded}

\begin{verbatim}
##      datos.revenue                             datos.genres
## 3211    2847246203 Action|Adventure|Fantasy|Science Fiction
## 5953    2797800564         Adventure|Science Fiction|Action
## 308     2187463944                            Drama|Romance
## 4948    2068223624 Action|Adventure|Science Fiction|Fantasy
## 5954    2046239637         Adventure|Action|Science Fiction
##                       datos.title
## 3211                       Avatar
## 5953            Avengers: Endgame
## 308                       Titanic
## 4948 Star Wars: The Force Awakens
## 5954       Avengers: Infinity War
\end{verbatim}

\end{document}
